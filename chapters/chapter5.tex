Améliorations
Work left to do

Those additions were done and tested on a x86 64-bit system using Linux, on a
closed project fork of the OCaml compiler, including memory profiling facilities.

debugging events propagation specific to the clambda middle end

improvements brought by the original developer concerns mostly a
new, middle end, inlining compiler pass called flambda.

infinity of programs with combination of constructs to test whether they play
out nicely together

%case of if\_const\_int.ml: goes back in the source to perform assignment to the
%b variable somewhat makes sense, although counter intuitive

global values bindings/vars are unmangled and see below

Enhancements

value inconsistencies
addition of .loc directives due to debugging events propagation the compiler is not aware of
the address ranges variables are available at

the available\_ranges pass is not exempt of bugs either, still experimental

User experience/interface minor:
- no function name symbol displayed in the backtrace
- global variables becomes available/displayed only after second call to `frame variable`

- value identifiers inserted at the lambda pass (bound, clos, index for loop,
match), not present in cmt/typedtree
- make it possible to "coerce"/typecast value?

Extensions
Work being done in same direction

\chapter{Conclusion\label{cha:chapter5}}
The final chapter summarizes the thesis. The first subsection outlines the main ideas behind Component X and recapitulates the work steps. Issues that remained unsolved are then described. Finally the potential of the proposed solution and future work is surveyed in an outlook.

\section{Summary\label{sec:summary}}

Explain what you did during the last 6 month on 1 or 2 pages!
\\
\\
\noindent The work done can be summarized into the following work steps

\begin{itemize}
		\item Analysis of available technologies
		\vspace{-0.11in}
		\item Selection of 3 relevant services for implementation
		\vspace{-0.11in}
		\item Design and implementation of X on Windows
		\vspace{-0.11in}
		\item Design and implementation of X on mobile devices
		\vspace{-0.11in}
		\item Documentation based on X
		\vspace{-0.11in}
		\item Evaluation of the proposed solution
\end{itemize}

\section{Dissemination\label{sec:dissemination}}

Who uses your component or who will use it?

Developers, OCaml users, system programmers
without needing knowledge of target architecture

Industry projects, EU projects, open source...? Is it integrated into a larger environment? Did you publish any papers?

\section{Problems Encountered\label{sec:problems}}

Summarize the main problems. How did you solve them? Why didn't you solve them?

\section{Outlook\label{sec:outlook}}
%\item Systèmes visés: Linux x86 64 bits
%\item Plupart des outils encore non disponible au public
% en particulier le fork du compilateur avec support pour profiling memoire
% devrait etre mis a disposition au public dans un futur plus ou moins proche
Emission DWARF \autocite{libmond} \autocite{dwpr}
may be available for OCaml 4.05, still a prototype


LLDB plugin reached the LLDB codebase,
release for 4.0

other patches still have to be integrated, may depend on the final form of the
debugging features

work done on ocp-lldb and ocplib-dwarf has been submitted
awaiting integration to typerex-binutils, reimplementation of the binutils tools
in OCaml and typerex-lldb

pull requests can be seen at

https://github.com/OCamlPro/typerex-lldb/pull/4
and
https://github.com/OCamlPro/typerex-binutils/pull/1

% compilateur reste a reste modifie en consequence
%\item ocp-lldb déjà disponible
%mais necessite fork avec memprof
%la branche sur laquelle je travaille necessite un autre fork encore avec mes ajouts

% Challenges et difficultes
%\item Augmenter le nombre de dbg events/locations
% chronophage car precision de methode naive + temps de compilation de la compiler suite
% comment faire composer les differents constructs ensemble
% Modifications entre debugger et compiler

ce que vous avez appris pendant votre stage que ce soit techniquement, méthodologiquement,
en termes de savoir-être en entreprise

I benefited in many ways from that internship:

I learned about inner workings of the OCaml toolchain,
I acquainted myself with multiple projects (OCaml, lldb) and contributed to their
codebases
I learned how debugging information is collected, synthetized and used in a debugger

learned to

weekly reporting of progress
time management between multiple related projects in parallel
compartimentation of related work in separate branchs using VCS
