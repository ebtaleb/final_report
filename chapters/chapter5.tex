Améliorations
Work left to do

Enhancements

User experience/interface minor:
- no function name symbol displayed in the backtrace
- global variables becomes available/displayed after second call to `frame variable`

Extensions
Work being done in same direction

\chapter{Conclusion\label{cha:chapter5}}
The final chapter summarizes the thesis. The first subsection outlines the main ideas behind Component X and recapitulates the work steps. Issues that remained unsolved are then described. Finally the potential of the proposed solution and future work is surveyed in an outlook.

\section{Summary\label{sec:summary}}

Explain what you did during the last 6 month on 1 or 2 pages!
\\
\\
\noindent The work done can be summarized into the following work steps

\begin{itemize}
		\item Analysis of available technologies
		\vspace{-0.11in}
		\item Selection of 3 relevant services for implementation
		\vspace{-0.11in}
		\item Design and implementation of X on Windows
		\vspace{-0.11in}
		\item Design and implementation of X on mobile devices
		\vspace{-0.11in}
		\item Documentation based on X
		\vspace{-0.11in}
		\item Evaluation of the proposed solution
\end{itemize}

\section{Dissemination\label{sec:dissemination}}

Who uses your component or who will use it?

Developers, OCaml users, system programmers
without needing knowledge of target architecture

Industry projects, EU projects, open source...? Is it integrated into a larger environment? Did you publish any papers?

\section{Problems Encountered\label{sec:problems}}

Summarize the main problems. How did you solve them? Why didn't you solve them?

\section{Outlook\label{sec:outlook}}
%\item Systèmes visés: Linux x86 64 bits
%\item Plupart des outils encore non disponible au public
% en particulier le fork du compilateur avec support pour profiling memoire
% devrait etre mis a disposition au public dans un futur plus ou moins proche
Emission DWARF \autocite{libmond} \autocite{dwpr}
may be available for OCaml 4.05, still a prototype

most of the improvements brought by the original developer concerns mostly a
new, middle end, inlining compiler pass called flambda


LLDB plugin reached the LLDB codebase,
release for 3.9.0 maybe?


%\item Plugin LLDB OCaml et ocp-dwarf peuvent être rendus disponible
% pour peu que la PR soit soumise
% mais modifications moins maintenables et touchant plus au debugger lui meme que le plugin
% compilateur reste a reste modifie en consequence
%\item ocp-lldb déjà disponible
%mais necessite fork avec memprof
%la branche sur laquelle je travaille necessite un autre fork encore avec mes ajouts

% Challenges et difficultes
%\item Augmenter le nombre de dbg events/locations
% chronophage car precision de methode naive + temps de compilation de la compiler suite
% comment faire composer les differents constructs ensemble
% Modifications entre debugger et compiler

ce que vous avez appris pendant votre stage que ce soit techniquement, méthodologiquement,
en termes de savoir-être en entreprise

I benefited in many ways from that internship:

inner workings of the ocaml toolchain
acquaintance with multiple codebases (lldb, ocaml)
implementation of a standardized file format (DWARF)
how debugging info is collected, synthetized and used in a debugger
contribution to open source software

learned to

weekly reporting of progress
time management between multiple related projects in parallel
compartimentation of related work in separate branchs using VCS
