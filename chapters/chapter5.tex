\chapter{Conclusion\label{cha:chapter5}}
The final chapter summarizes the thesis. The first subsection outlines the main ideas behind Component X and recapitulates the work steps. Issues that remained unsolved are then described. Finally the potential of the proposed solution and future work is surveyed in an outlook.

\section{Summary\label{sec:summary}}

Explain what you did during the last 6 month on 1 or 2 pages!
\\
\\
\noindent The work done can be summarized into the following work steps:

\begin{itemize}
		\item Propagation of debugging events through the backend
		\vspace{-0.11in}
		\item Integration of the DWARF emitter and fixes
		\vspace{-0.11in}
		\item Implementation of the DWARF reading library
		\vspace{-0.11in}
		\item Implementation and integration of a OCaml language plugin for LLDB
		\vspace{-0.11in}
		\item Additions to the native debugger ocp-lldb
\end{itemize}

\section{Outlook\label{sec:outlook}}

Those additions were done and tested on a x86 64-bit system using Linux, on a
closed project fork of the OCaml compiler version 4.02.1, including memory
profiling facilities, debugging events propagation specific to the clambda
middle end and type information inclusion into binaries.
It should be released in the near future.

The OCaml language plugin reached the LLDB codebase,
release for 4.0

Other remaining patches still have to be integrated into LLDB, although it
may depend on the final form of the debugging features.

The work done on ocp-lldb and ocplib-dwarf has been submitted and is now awaiting integration
to typerex-binutils, reimplementation of the binutils tools in OCaml and typerex-lldb

Pull requests can be seen
there\footnote{\url{http://github.com/OCamlPro/typerex-lldb/pull/4}} \footnote{\url{http://github.com/OCamlPro/typerex-binutils/pull/1}}

\subsection{Areas of improvements}


\subsection{Remaining work}

%\item Augmenter le nombre de dbg events/locations

\subsection{Extensions}

- make it possible to "coerce"/typecast value of identifiers inserted at the
lambda pass and that the typed AST, nor the symbol table it is based on are not
aware of.

\subsection{Related works}

This work\footnote{\url{http://github.com/ocaml/ocaml/pull/247}} makes backtraces aware
of inline function calls (by creating fake stack frames?)
It could be possible to express the inlined function attribute in DWARF data for
inlined functions to aid the debugger.

Emission DWARF \footnote{\url{http://github.com/ocaml/ocaml/pull/574}}
may be available for OCaml 4.05, still a prototype

Improvements brought by the original developer concerns mostly a
new, middle end, inlining compiler pass called flambda.
notable additions includes submodule inclusion in module paths and addition of
phantom lets for access to constant values even when those values have been
folded and are not present in the machine code.

libmonda\footnote{\url{http://github.com/mshinwell/libmonda}} aims to be a
debugger-independent, OCaml debugging layer library for helping with
interpretation of values at runtime.
Internal compiler data structures might change from one version to another,
hence backwards compatibilty with prior versions should be kept in mind

\section{Dissemination\label{sec:dissemination}}

Who uses your component or who will use it?

OCaml users, system programmers
Developers without needing knowledge of target architecture

\section{Problems Encountered\label{sec:problems}}

Summarize the main problems. How did you solve them? Why didn't you solve them?

value inconsistencies
addition of .loc directives due to debugging events propagation the compiler is not aware of
the address ranges variables are available at

the available\_ranges pass is not exempt of bugs either, still experimental

% chronophage car precision de methode naive + temps de compilation de la compiler suite
% comment faire composer les differents constructs ensemble

User experience/interface minor:
- no function name symbol displayed in the backtrace
- global variables becomes available/displayed only after second call to `frame variable`

- value identifiers inserted at the lambda pass (bound, clos, index for loop,
match), not present in cmt/typedtree

infinity of programs with combination of constructs to test whether they play
out nicely together

%case of if\_const\_int.ml: goes back in the source to perform assignment to the
%b variable somewhat makes sense, although counter intuitive

global values bindings/vars are unmangled and see below



\subsection{Lessons drawn from the internship}

I benefited in many ways from that internship:

I learned about inner workings of the OCaml toolchain,
I acquainted myself with multiple projects (OCaml, lldb) and contributed to their
codebases
I learned how debugging information is collected, synthetized and used in a debugger

learned to

weekly reporting of progress
time management between multiple related projects in parallel
compartimentation of the different projects in separate branchs using version control
software
