\chapter{Introduction\label{cha:chapter1}}

There are things to be aware of when dealing with \gls{native}:
\begin{itemize}
    \item it does not retain any information about
        the abstractions of the source programming language in the original
        source code, because the CPU does not need them to run the program,
%backend compiler passes
%reorganize and reposition instructions
    \item compiler \glspl{optim} can move, add and transform code
        in such a way that it might not be possible to identify which
        source code statement corresponds to a particular set of instructions.
\end{itemize}

Hence, compilation of a program to native code involve a huge loss of information,
information that is usually collected by compilers and bundled with binaries and
\glspl{objectfile} in the form of debugging information for debuggers.

\section{Motivation\label{sec:moti}}

There are three main reasons for this work :

\begin{itemize}
    \item Cases where it is necessary to debug native code may arise, e.g a bug
        appearing only in an optimized version of a program.
    \item For now, there is no unified solution for debugging OCaml native
        code on any platform. The bare minimum is here (\gls{backtrace} support,
        partial step by step into function entry points and function calls),
        but mainstream debuggers such as gdb and lldb remain largely unaware that they
        might deal with OCaml native programs.
    \item In that situation, inspecting disassembled code instruction by
        instruction and raw memory becomes a tedious task in a large program,
        that requires knowledge of the target architecture and the OCaml native
        runtime, hence the need for a source-level native debugger.
\end{itemize}

\newpage

\section{Objectives and scope\label{sec:objective}}

The work presented here aims to improve debugging experience concerning OCaml
native compiled code. \\
It involves coordination between the compiler and the debugger:

\begin{itemize}
    \item Modifying the OCaml native compiler to output more debugging
        information,
        %such as line information (mapping between lines in source
        %code and machine code addresses) and runtime location of variables,
        and storing that information in a format that a debugger can make
        sense of.
    \item  Improving an OCaml native debugger prototype, written in OCaml and using LLDB
        API bindings, and implementing the main features found in a source-level
        debugger (setting breakpoints, proper symbol handling, stepping into the source
        code, printing values of variables).
\end{itemize}

\section{About OCamlPro}

The OCamlPro company does mainly research and development and provides
consulting services for the purpose of developing and promoting OCaml in the
software development industry.
Notable products/highlights include OCaml online learning tools, contributions
to the Open Source Software community, in particular additions to the OCaml
compiler suite and the SMT solver Alt-Ergo.

%ne consacrez pas plus d’une page
%à la description de l’entreprise et du contexte du stage

%Internship context

%Intern bound to research and development

%Internship timeline (every forthnight)

\section{Outline\label{sec:outline}}

\textbf{Chapter \ref{cha:chapter2}} will give a overview of the DWARF
data debugging format, its structure and related work around it.
\\
\\
\textbf{Chapter \ref{cha:chapter3}}, we will have a look at the process of
compilation to native code and additions done to that process.
\\
\\
\textbf{Chapter \ref{cha:chapter4}} will tackle the OCaml runtime management of
values will be explained, introduces to LLDB, related OCaml binding ocplib-lldb and
the native debugger ocp-lldb.
\\
\\
\textbf{Chapter \ref{cha:chapter5}} will summarize the thesis,
discuss the problems that occurred and possible future work.
