\chapter{Introduction to the DWARF format\label{cha:chapter2}}

To establish correspondence between high-level source code and low-level machine code, one wants to
\begin{itemize}
    \item  collect all information generated im the compiling process
    \item  and represent that information in a suitable format usable by debuggers, concisely and compactly
\end{itemize}

DWARF is a debugging data format allowing support for source level debugging
it has with the following characteristics:

It is widely used today
it is language agnostic, independent of compiler tooling (linker/assembler),
target architecture and debuggers.
It is somewhat object file agnostic, although mostly associated with the ELF object file format.

it is standardized, meaning it has specifications and norms codified into
a formal document.
A commitee oversees additions/extensions to the standard to follow the
appearance of programming languages and their evolution

\section{Important DWARF sections and their contents}

%\begin{description}[leftmargin=!,labelwidth=\widthof{\bfseries .debug\_abbrev}]
\begin{description}[labelwidth=\widthof{\bfseries .debug\_abbrev},align=parright]
    \item[.debug\_abbrev] Abbreviations used to decode the .debug\_info section
    \item[.debug\_frame] Call frame information table
    \item[.debug\_info] Core DWARF data section
    \item[.debug\_line] Line number information table
    \item[.debug\_loc] Location lists for runtime location of variables/parameters
    \item[.debug\_str] String table reference in .debug\_info
\end{description}

\section{Storage}

The debugging information is stored in sections in the object file.

In the separate compilation process, the DWARF sections of each object file are
put together in their common corresponding section in the final binary

Binary data format

Multiple techniques are used to reduce the size of the debugging information:

LEB128 is a variable length compression encoding
it allows for encoding of signed and unsigned integers of arbitrary size.
Used in DWARF to encode values of various fields (like attributes)

Some of the section tables can become quite large.
In order to save space, those tables are encoded as bytecode instructions to be executed by state machines to obtain their fully expanded form

\section{Structure}

\subsection{Debugging information entry}

In DWARF, the base entity manipulated is called an debugging information entry, or DIE.
A DIE is made of a tag specifying what language construct it describes and a list of attributes giving more details concerning that construct.

A DIE tag can designate for example entities such as:
\begin{description}
    \item[DW\_TAG\_subprogram] — A subroutine/function
    \item[DW\_TAG\_variable] — A variable
    \item[DW\_TAG\_formal\_parameter] — An argument within function parameters
    \item[DW\_TAG\_lexical\_block] - Define lexical scoping of local variables
    \item[DW\_TAG\_base\_type] - Primitive type not defined in terms of other
        types
\end{description}

An attribute is a name/value couple.
The name attribute indicates what the value represents (name, absolute address,
bytecode, offset, string, integer) and what class of values are supported
It is hence possible for attributes values to reference other DIEs and DWARF sections.

\begin{description}
    \item[DW\_AT\_name] String naming the designated DIE
    \item[DW\_AT\_stmt\_list] Offset referencing a location list
    \item[DW\_AT\_type] Reference to a type DIE
    \item[DW\_AT\_low\_pc] address of the first machine instruction
    \item[DW\_AT\_high\_pc] address or offset/constant of the first location past the last instruction
\end{description}

A DIE can be nested in another parent DIE, have siblings and children,
can be represented as a tree with an arbitrary number of children

The subprogram DIE owns DIEs describing the subprogram.

\subsection{Compilation unit}

It designates two entities :

each separately compiled source file is considered as a compilation unit
the root of a DIE tree that starts the DWARF data for the source file it represents
It contains general information about the compilation,
such as the programming language, the compiler version used, source file name, flags

\section{Example}

\section{The ocplib-dwarf library}

Working with DWARF programmatically
it is a library written in OCaml that reads and prints DWARF data in a human readable format similar to objdump
made in a effort to learn and comprehend the DWARF format during the internship.

DWARF emission writing and editing features are not yet implemented
not sure it is useful, as this task is left to compilers

high level desc

%todo prepare a schema

driver with command line option handling

offers abstraction that allow DWARF data manipulation
can visualize the DIE tree structure in debug\_info in DOT format using graphviz
to emulate mutable trees, a zipper library was implemented

utility functions for reading binary data

section parsing implemented accordingly to the DWARF4 standard by its own module
printing module operates on DWARF data structures in ocaml

hexadecimal view available for debugging purposes

