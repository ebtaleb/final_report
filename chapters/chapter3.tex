\chapter{The OCaml native compiler}

The OCaml language support multiple programming paradigms

functional programming style, pure computations, math functions (always evals to same value acr
different executions for same args),

in contrast, imperative programming can change directly state of the prog,
resulting in side effects where return value of a routine may change acr mult
execution

stepping through function call is fine, regardless of whether they are pure or
not
support assignments and mutable variables

map filter, first class functions taking other functions as arguments, partial
application it may not make much sense to debug this the same way one would with
an imperative language such as C

\section{Compilation process}

The OCaml native toolchain

ocamlc Bytecode version of the bytecode compiler
ocamlc.opt Native code version of the bytecode compiler
ocamlopt Bytecode version of the native compiler
ocamlopt.opt Native code version of the native compiler

the source program goes though multiple passes/transformations

among those steps, all passes up to lambda are common to both bytecode and
native compilers, the next ones are specific to the native compiler

lexing / lexical analysis
split the program from a sequence of characters into a sequence of tokens

parsetree / syntactic analysis - parsing

check whether the program as a seq of tokens is gramatically valid, and if so
construct a base in-memory representation of the program, an abstract syntax
tree
contain file and line location information

typedtree - typing - semantic analysis
annotate AST with type information
type inference, pattern matching analysis?

lambda
AST with pattern matching compilation to if and switch constructs
loss of most of location information, except for debugging purposes

clambda
closure conversion, inlining, constant propagation

cmm
primitives conversion, constant folding

mach
register liveness analysis, register allocation

linearize
code as list/seq of pseudo-instructions, close to machine code
with calls and jumps

codegen

% todo : make figure with all different compilation steps

\begin{algorithmic}[1]
	\If{some condition is true}
	\State do some processing
	\ElsIf{some other condition is true}
	\State do some different processing
	\ElsIf{some even more bizarre condition is met}
	\State do something else
	\Else
	\State do the default actions
	\EndIf
\end{algorithmic}

\section{Additions to the compiler}

Concerning the modifications done to the native compiler

scope limiting to x86 64-bits systems using Linux
closed project fork of the ocaml compiler,
including memory profiling facilities

libraries in compilerlibs allows use of the lexer and parser facilities, esp
useful when operating over parsetree/typedtree representation

used to build symbol table mapping identifiers in the prog to their type
for runtime value interpretation

\subsection{Location information}

Definition of debugging events

from the ocaml manual
%http://caml.inria.fr/pub/docs/manual-ocaml-4.02-1/debugger.html#sec350

Events are “interesting” locations in the source code, corresponding to the beginning or end of evaluation of “interesting” sub-expressions. Events are the unit of single-stepping (stepping goes to the next or previous event encountered in the program execution). Also, breakpoints can only be set at events. Thus, events play the role of line numbers in debuggers for conventional languages.

those events are here used in the bytecode with the bytecode, source-level debugger ocamldebug

it was stated previously that among the debugging information to be added, line
information is important, both for stepping into the source program statement by
statement, but also for setting breakpoints by file and line number

the idea here is to propagate the location information contained in those events
through the native code backend

by wrapping all AST/IR nodes into a struct containing the node and a field with
location information from the debugging event

this has been done for every native backend pass all the way up to linearize (that already
contains such a field)

then attached location information to some nodes/constructs in particular

integers - as empty lists, None represented as integer 1/0
setfield primitive - responsible for mut assignments
pro cons

\subsection{Runtime location of variables}

dwarf - available\_regs/ranges
types
for each, what effect is achieved, tradeoffs
what/why/how

\subsection{DWARF emitter}

\subsection{Type information}



