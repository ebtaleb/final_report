, what/why/how
The OCaml language?
The OCaml native toolchain
hl architecture, compilation process

debug loc
dwarf - available\_regs/ranges
types
for each, what effect is achieved, tradeoffs


\chapter{Requirements\label{cha:chapter3}}
This section determines the requirements necessary for X. This includes the functional aspects, namely Y and Z, and the non functional aspects such as A and B.

\section{Overview\label{sec:reqoverview}}

In this chapter you will describe the requirements for your component. Try to group the requirements into subsections such as 'technical requirements', 'functional requirements', 'social requirements' or something like this. If your component consist of different partial components you can also group the requirements for the corresponding parts.

Explain the source of the requirements.

Example: The requirements for an X have been widely investigated by Organization Y.

In his paper about Z, Mister X outlines the following requirements for a Component X.

\section{Technical Requirements\label{sec:techreq}}

The following subsection outlines the technical requirements to Component X.

\begin{algorithmic}[1]
	\If{some condition is true}
	\State do some processing
	\ElsIf{some other condition is true}
	\State do some different processing
	\ElsIf{some even more bizarre condition is met}
	\State do something else
	\Else
	\State do the default actions
	\EndIf
\end{algorithmic}

\subsection{Sub-component A\label{sec:reqsuba}}

\textbf{Interoperability}
\\
Lorem Ipsum...
\\
\\
\textbf{Scalability}
\\
Lorem Ipsum...

\subsection{Sub-component B\label{sec:reqsubb}}

Lorem Ipsum...

\section{Social Requirements\label{sec:socreq}}

Component X must compete with Y. Hence, it is required to provide an excellent usability. This includes ...
