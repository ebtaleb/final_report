\newglossaryentry{api}{name={API},
    description={An application programming interface is a set of code, protocols and specifications used to build software and for communication between software components}}
    %%api is source-code based, whereas abi is binary code based

\newglossaryentry{machine}{name={machine code},
    description={set of binary code instructions that can be executed directly by a computer's CPU}}

\newglossaryentry{native}{name={native code}, description={See \gls{machine}}}

\newglossaryentry{ir}{name={IR},
    description={An intermediary representation is a machine, language
    independent version of a program's source code manipulated by the compiler
    inbetween source and target languages},
    first={intermediary representation (IR)}}

\newglossaryentry{optim}{name={optimization},
    description={transformation technique of computer code to make it more efficient (CPU time, memory consumption) while preserving correctness and side effects}}

\newglossaryentry{backtrace}{name={backtrace},
    description={listing of active function calls at a certain point in a program's execution (with the most recent one at the top)}}

\newglossaryentry{inlin}{name={inlining},
    description={optimization replacing a function call site with its body}}

\newglossaryentry{objectfile}{name={object file},
    description={compiler output file consisting of machine code, linking and debugging metadata (symbols, stack unwinding information, profiling information), not directly executable}}

\newglossaryentry{runtime}{name={runtime system},
    description={software code supporting the execution of a program written in
    a programming language, required to implement features of the language.\\
        E.g, ocamlrun, used to execute the bytecode program, can be considered as part
        of the OCaml runtime, as it handles bytecode interpretation, memory allocation
        and garbage collection.}}

\newglossaryentry{binding}{name={binding},
    description={library providing an interface to another one written in a different programming language}}

\newglossaryentry{boxed}{name={boxed value},
    description={wrapper structure around a primitive type allocated on the heap
    memory}}

\newglossaryentry{bb}{name={basic block},
    description={sequence of statements without branching statements among them}}

\newglossaryentry{cf}{name={control flow},
    description={order of execution/evaluation of individual statements,
    by branching, loop, function call and return statements.}}

\newglossaryentry{cfg}{name={control flow graph},
    description={directed graph whose vertices are \glspl{bb} and edges
    possible transfer of \gls{cf} from one basic block to
    another. It allows modeling of all possible executions of a program.},
    first={control flow graph (CFG)}}

\newglossaryentry{ffi}{name={FFI},
    description={A foreign function interface is a mechanism enabling cooperation of programs and functions
    written in different programming languages, e.g OCaml functions called from
    C programs or the other way around},
    first={foreign function interface (FFI)}}

\newglossaryentry{ast}{name={abstract syntax tree},
    description={tree representation of a program's source code and syntactical structure}}

\newglossaryentry{cfi}{name={CFI},
    description={Call Frame Information. DWARF can record information
    about where frames are located on the function call stack. This allows for
    backtrace reconstruction and exception handling.},
    first={call frame information (CFI)}
    %,long={call frame information}
    }

\newglossaryentry{cfa}{name={CFA},
    description={The Canonical Frame Address is the value of the stack pointer at the call
    site in the previous frame.},
    first={canonical frame address (CFA)},
    }

%\newglossaryentry{}{name={},
    %description={}}

