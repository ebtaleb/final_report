%!TEX root = ../dokumentation.tex

%
% vorher in Konsole folgendes aufrufen:
%	makeglossaries makeglossaries dokumentation.acn && makeglossaries dokumentation.glo
%

%
% Glossareintraege --> referenz, name, beschreibung
% Aufruf mit \gls{...}
%

%\newacronym{cd}{CD}{compact disk}

\newglossaryentry{Glossareintrag}{name={Glossareintrag},plural={Glossareinträge},description={Ein Glossar beschreibt verschiedenste Dinge in kurzen Worten}}

\newglossaryentry{computer}
{
    name=computer,
        description={is a programmable machine that receives input,
            stores and manipulates data, and provides
                output in a useful format}
}

\newglossaryentry{potato}{name={potato},plural={potatoes},
description={starchy tuber}}
\newglossaryentry{cabbage}{name={cabbage},
description={vegetable with thick green or purple leaves}}
\newglossaryentry{carrot}{name={carrot},
description={orange root}}

%\newacronym{api}{API}{Application Programming Interface }

%%% The glossary entry the acronym links to   
\newglossaryentry{api}{name={API},
    description={An Application Programming Interface (API) is a particular set
    of rules and specifications that a software program can follow to access and
    make use of the services and resources provided by another particular software
    program that implements that API}}

    %%% define the acronym and use the see= option
    %\newglossaryentry{api}{type=\acronymtype, name={API}, description={Application
    %Programming Interface}, first={Application
    %Programming Interface (API)\glsadd{apig}}, see=[Glossary:]{apig}}
